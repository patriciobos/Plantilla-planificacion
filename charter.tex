\documentclass[
11pt, % The default document font size, options: 10pt, 11pt, 12pt
codirector, % Uncomment to add a codirector to the title page
]{charter} 




% El títulos de la memoria, se usa en la carátula y se puede usar el cualquier lugar del documento con el comando \ttitle
\titulo{Sistema de riego y control de huertas} 

% Nombre del posgrado, se usa en la carátula y se puede usar el cualquier lugar del documento con el comando \degreename
%\posgrado{Carrera de Especialización en Sistemas Embebidos} 
\posgrado{Carrera de Especialización en Internet de las Cosas} 
%\posgrado{Carrera de Especialización en Intelegencia Artificial}
%\posgrado{Maestría en Sistemas Embebidos} 
%\posgrado{Maestría en Internet de las cosas}

% Tu nombre, se puede usar el cualquier lugar del documento con el comando \authorname
\autor{Lic. Siciliano Gustavo Hernan} 

% El nombre del director y co-director, se puede usar el cualquier lugar del documento con el comando \supname y \cosupname y \pertesupname y \pertecosupname
\director{Director a definir}
\pertenenciaDirector{FIUBA} 
% FIXME:NO IMPLEMENTADO EL CODIRECTOR ni su pertenencia
\codirector{Codirector a definir} % para que aparezca en la portada se debe descomentar la opción codirector en el documentclass
\pertenenciaCoDirector{FIUBA}

% Nombre del cliente, quien va a aprobar los resultados del proyecto, se puede usar con el comando \clientename y \empclientename
\cliente{Alejandra Vranic}
\empresaCliente{Universidad Nacional de Lanús}

% Nombre y pertenencia de los jurados, se pueden usar el cualquier lugar del documento con el comando \jurunoname, \jurdosname y \jurtresname y \perteunoname, \pertedosname y \pertetresname.
\juradoUno{Jurado a definir}
\pertenenciaJurUno{FIUBA} 
\juradoDos{Jurado a definir}
\pertenenciaJurDos{FIUBA}
\juradoTres{Jurado a definir}
\pertenenciaJurTres{FIUBA}
 
\fechaINICIO{20 de Octubre de 2022}		%Fecha de inicio de la cursada de GdP \fechaInicioName
\fechaFINALPlan{8 de Diciembre de 2022} 	%Fecha de final de cursada de GdP
\fechaFINALTrabajo{a definir}	%Fecha de defensa pública del trabajo final


\begin{document}

\maketitle
\thispagestyle{empty}
\pagebreak


\thispagestyle{empty}
{\setlength{\parskip}{0pt}
\tableofcontents{}
}
\pagebreak


\section*{Registros de cambios}
\label{sec:registro}


\begin{table}[ht]
\label{tab:registro}
\centering
\begin{tabularx}{\linewidth}{@{}|c|X|c|@{}}
\hline
\rowcolor[HTML]{C0C0C0} 
Revisión & \multicolumn{1}{c|}{\cellcolor[HTML]{C0C0C0}Detalles de los cambios realizados} & Fecha      \\ \hline
0      & Creación del documento                                 &\fechaInicioName \\ \hline
1      & Se completa hasta la sección 5 inclusive                 & 2 de Noviembre de 2022 \\ \hline
2      & Se aplican las correcciones de la primer entrega                 & 6 de Noviembre de 2022 \\ \hline
%2      & Se completa hasta el punto 7 inclusive
%		  Se puede agregar algo más \newline
%		  En distintas líneas \newline
%		  Así                                                    & dd/mm/aaaa \\ \hline
%3      & Se completa hasta el punto 11 inclusive                & dd/mm/aaaa \\ \hline
%4      & Se completa el plan	                                 & dd/mm/aaaa \\ \hline
\end{tabularx}
\end{table}

\pagebreak



\section*{Acta de constitución del proyecto}
\label{sec:acta}

\begin{flushright}
Buenos Aires, \fechaInicioName
\end{flushright}

\vspace{2cm}

Por medio de la presente se acuerda con el Lic. \authorname\hspace{1px} que su Trabajo Final de la \degreename\hspace{1px} se titulará ``\ttitle'', el cual consistirá esencialmente en la implementación de un prototipo de monitoreo de humedad y riego automático de huertas. Tendrá un presupuesto preliminar estimado de {600} hs de trabajo y {\$100.000 pesos argentinos} para la compra de materiales, con fecha de inicio \fechaInicioName\hspace{1px} y fecha de presentación pública \fechaFinalName.

Se adjunta a esta acta la planificación inicial.

\vfill

% Esta parte se construye sola con la información que hayan cargado en el preámbulo del documento y no debe modificarla
\begin{table}[ht]
\centering
\begin{tabular}{ccc}
\begin{tabular}[c]{@{}c@{}}Dr. Ing. Ariel Lutenberg \\ Director posgrado FIUBA\end{tabular} & \hspace{2cm} & \begin{tabular}[c]{@{}c@{}}Mg. \clientename \\ \empclientename \end{tabular} \vspace{2.5cm} \\ 
\multicolumn{3}{c}{\begin{tabular}[c]{@{}c@{}} \supname \\ Director del Trabajo Final\end{tabular}} \vspace{2.5cm} \\
%\begin{tabular}[c]{@{}c@{}}\jurunoname \\ Jurado del Trabajo Final\end{tabular}     &  & \begin{tabular}[c]{@{}c@{}}\jurdosname\\ Jurado del Trabajo Final\end{tabular}  \vspace{2.5cm}  \\
%\multicolumn{3}{c}{\begin{tabular}[c]{@{}c@{}} \jurtresname\\ Jurado del Trabajo Final\end{tabular}} \vspace{.5cm}                                                                     
\end{tabular}
\end{table}




\section{1. Descripción técnica-conceptual del proyecto a realizar}
\label{sec:descripcion}


\begin{consigna}{black} % El bloque "consigna" se usa para poner texto en rojo y dar una pequeña ayuda sobre cómo completar la sección

El presente trabajo nace a partir de un proyecto de investigación originado en la UNLa (Universidad Nacional de Lanús) y consiste en el desarrollo de un prototipo tecnológico que pueda medir el porcentaje de humedad de un conjunto de plantas en una huerta. Además, se implementará un software de seguimiento y control de las mediciones.

La finalidad del proyecto es contar con un sistema de monitoreo de huertas que colabore con mantener condiciones optimas para los cultivos. Esto se logrará tanto con revisiones manuales de las métricas obtenidas, como con el sistema de riego automático. Este último activará una válvula de agua que conseguir un buen cuidado de las plantas.
En base a esto se proyecta que una huerta va a estar compuesta por varios sectores. En cada uno se van a agrupar plantas con características de cuidado similares. Además, cada sector contará con un prototipo con una placa ESP32 y dos sensores para medir el porcentaje de humedad del suelo y del ambiente. Las placas deberán tener conexión Wi-Fi para enviar las métricas al Backend del servidor vía protocolo HTTP. Ese sistema va a nutrir una plataforma Web. En el Frontend habrá un panel de administración por sector, para que en cada uno se puedan fijar valores mínimos de porcentaje de humedad. En caso de que las mediciones sean inferiores a estos valores de control, el Backend generará un mensaje para los dispositivos. De esta manera se accionará la manguera de riego y se podrá tener un cuidado simple y automático de las plantas.


%\vspace{25px}

\begin{figure}[htpb]
\centering 
\includegraphics[width=.9\textwidth]{./Figuras/diagBloques.png}
\caption{Diagrama del sistema.}
\label{fig:diagBloques}
\end{figure}

\vspace{25px}

\end{consigna}


\section{2. Identificación y análisis de los interesados}
\label{sec:interesados}

\begin{consigna}{black} 

\begin{table}[ht]
%\caption{Identificación de los interesados}
%\label{tab:interesados}
\begin{tabularx}{\linewidth}{@{}|l|X|X|l|@{}}
\hline
\rowcolor[HTML]{C0C0C0} 
Rol&
Nombre y Apellido&
Organización&
Puesto\\ \hline

Auspiciante&
\clientename &
\empclientename &
Directora de la Licenciatura en Sistemas\\ \hline

Cliente&
\clientename &
\empclientename & 
Directora de la Licenciatura en Sistemas\\ \hline

Impulsor&
Laura Loidi&
\empclientename &
Coordinadora de proyectos de investigación\\ \hline

Responsable&
\authorname &
FIUBA&
Alumno\\ \hline

Colaboradores&
Leandro Rios&
\empclientename &
Docente investigador\\ \hline

Orientador&
\supname &
\pertesupname &
Director de trabajo final\\ \hline

Equipo&
-Reboredo Damian\newline 
-Contento Guido\newline
-Otegui Luciano&
\empclientename &
Estudiantes investigadores\\ \hline

Usuario final&
Estudiantes UNLa y Vallese&
\empclientename &
Estudiantes\\ \hline
\end{tabularx}
\end{table}


\begin{itemize} 
	\item Cliente: Alenadra Vranic (que también oficia como Auspiciante) necesita tener un plan de trabajo para Diciembre 2022 y avances trimestrales duramente 2023.
	\item Impulsor: Laura Loidi en su rol de coordinadora de proyectos de investigación dará asistencia y seguimiento metodológico.	
	\item Colaborador: Leandro Rios va a dar soporte para la compra, armado e instalación de los dispositivos en la huerta.
	\item Equipo: los estudiantes finalizarán sus tareas de investigación entre diciembre del 2022 y febrero del 2023. Se tiene que delimitar el alcance y las futuras líneas de desarrollo para este trabajo para fines de noviembre del 2022.
	\item Orientador: aún se definió el director del trabajo. Esto tiene que resolverse antes de la quinta clase de Gestión de Proyectos.
\end{itemize}

\end{consigna}



\section{3. Propósito del proyecto}
\label{sec:proposito}

\begin{consigna}{black}

El propósito de este proyecto es generar un rédito intelectual, económico y cultural dentro de la UNLa. Para cumplir esto se tienen 4 verticales principales de trabajo:

1- Armar un prototipo técnico para tomar mediciones de las plantas en una huerta. Con esos valores se deberá operar una manguera de agua y brindar métricas del estado de los cultivos.

2- Desarrollar un software web que pueda tomar las métricas del prototipo y gestionarlas a través de un panel de control.

3- Implementar en la escuela oficios de Felipe Vallese (http://oficios.unla.edu.ar), y en la UNLa, una serie de huertas que sirvan para dar empleo y nutrir de alimentos estas organizaciones y a escuelas aledañas.

4- Consolidar un curso que conste de capacitaciones a darse en Felipe Vallese. La idea es extender el conocimiento de cómo montar el prototipo para personas de la comunidad.
Como un punto extra, el material generado en el proyecto también se podría utilizar para incluir una materia optativa en la carrera de Licenciatura en Sistemas de la UNLa.


\end{consigna}

\section{4. Alcance del proyecto}
\label{sec:alcance}

\begin{consigna}{black}

El presente proyecto tiene como alcance trabajar en las verticales 1, 2 y 3 de la sección anterior.

En relación al punto 1, se completará el prototipo inicial de la placa ESP32 con sus dos sensores. Este trabajo se está llevando adelante con el estudiante Damian Reboredo. Actualmente ya se cuenta con una versión funcional que no contempla a la manguera con una conexión de agua. Se deberá finalizar esa tarea, más la integración con el sistema de control y monitoreo.

%\vspace{25px}

\begin{figure}[htpb]
\centering 
\includegraphics[width=.7\textwidth]{./Figuras/diagDipos.png}
\caption{Diagrama del prototipo.}
\label{fig:diagDipos}
\end{figure}

\vspace{25px}

\end{consigna}

\begin{consigna}{black}

Con respecto al punto 2, se trabajará en torno primer entregable del sistema de control y monitoreo. Este desarrollo se está llevando adelante con los estudiantes Guido Contento y Luciano Otegui. Dicho equipo tiene como foco finalizar la estructura base del Backend y Frontend con las altas, bajas y modificaciones de las secciones iniciales del sistema. También se realizará la primera propuesta de integración con los dispositivos. El resto de las funcionalidades quedará en el marco de trabajo del proyecto.

Finalmente, sobre el punto 3, se quiere montar una huerta en la escuela de oficios Felipe Vallese. La expectativa es que cuente con tres sectores que tengan los dispositivos de medición. Para lograr esto se debe completar un plano de la huerta. En él, se diagramará la distribución de los sectores y la ubicación de los dispositivos. Esto último, más la adquisición de los materiales, forma parte del alcance de la presente documentación.

%\vspace{25px}

\begin{figure}[htpb]
\centering 
\includegraphics[width=.7\textwidth]{./Figuras/diagHuerta.png}
\caption{Plano alto nivel la huerta.}
\label{fig:diagHuerta}
\end{figure}

\vspace{25px}

No forma parte del alcance el armado en sí de la huerta, ni de su cableado eléctrico y tuberías de agua. Esta parte del trabajo la realizarán los empleados de Felipe Vallese.

\end{consigna}


\section{5. Supuestos del proyecto}
\label{sec:supuestos}

\begin{consigna}{black}
Para el desarrollo del presente proyecto se supone que:

\begin{itemize}
	\item 1: no habrá inconvenientes con el armado de la huerta por parte del equipo de Felipe Vallese.
	\item 2: las condiciones de acceso a la red de electricidad y agua serán correctas.
	\item 3: la conectividad Wi-Fi será viable en el espacio de la huerta.
	\item 4: se podrá contar con el presupuesto acordado para la compra de materiales e insumos.
	\item 5: este proyecto seguirá formando parte los trabajos de investigación prioritarios para la UNLa, por lo menos hasta fines del año 2023.
\end{itemize}
\end{consigna}

\section{6. Requerimientos}
\label{sec:requerimientos}

Los requerimientos del proyecto están agrupados por afinidad:

\begin{enumerate}
	\item Requerimientos del dispositivo
		\begin{enumerate}
			\item Debe tener un código interno para ser identificado unívocamente en el software de control.
			\item Debe contar con una placa ESP32 más dos sensores de humedad (uno de ambiente y otro de suelo).
			\item La placa ESP32 debe usar los sensores para medir los porcentajes correspondientes de las plantas de su sector. 
			\item Debe centralizar las métricas obtenidas y contar con Wi-Fi para trasmitirlas a un servidor. 
			\item Tiene que poder activar la apertura o cierre de una válvula de agua usando un comando interno.
		\end{enumerate}
	
	\item Requerimientos de integración del backend del software
		\begin{enumerate}
			\item Debe contar con un endpoint rest API para consultar las mediciones de un dispositivo.
			\item Debe contar con un endpoint rest API para solicitar a un dispositivo la apertura de la válvula de agua durante 30 segundos.
			\end{enumerate}

	\item Requerimientos del sistema para los usuarios
		\begin{enumerate}
			\item Debe permitir a un usuario loguearse al sistema usando su mail y constraseña.
			\item Debe permitir a un usuario desloguearse del sistema.
			\item Debe permitir a un usuario recuperar y cambiar su contraseña.
			\item Debe tener una sección de visualización y modificación de los datos de perfil del usuario logueado.
			\item Debe contemplar permisos para cada rol del sistema:
			\begin{enumerate}
			\item Administrador: acceso todas las funcionalidades.
			\item Responsable: acceso a las funcionalidades de administración de sus huertas.
			\item Visitante: acceso a las funcionalidades de visualización de sus huertas asociadas.
			\end{enumerate}
			\item Debe tener una sección para ver, modificar y eliminar huertas del sistema.
			\item Debe tener una sección para ver, modificar y eliminar usuarios del sistema.
			\item Debe tener una sección para administrar huertas vinculando el código de un dispositivo por sector.
			\item Debe tener una sección para administrar los porcentajes de aceptación de humedad por sector.
			\item Debe tener una sección para visualizar las mediciones de un dispositivo por sector.
			\item Debe tener una sección para ver, modificar y eliminar usuarios con rol visitante para una huertas.
			\item Debe tener una sección para ver, modificar y eliminar roles del sistema.
			\item Debe tener una sección para asociar un rol a un usuario nuevo.
			\item Debe permitir que un usuario (con rol responsable) solicite su alta en el sistema usando mail y password.
		\end{enumerate}		
		
	\item Requerimientos no funcionales
		\begin{enumerate}
			\item El sistema deberá tener un usuario administrador por defecto.
			\item Cada dispositivo deberá instalarse en un sector correctamente delimitado de la huerta.
			\item La huerta va a contar con sectores con platas que tengan características de cuidado similares.
		\end{enumerate}
		
\end{enumerate}

\section{7. Historias de usuarios (\textit{Product backlog})}
\label{sec:backlog}

Ponderación: se utiliza la sucesión de Fibonacci.
\newline Priorización: se puntúa usando una prioridad de 1 (alta) a 5 (baja).

\begin{itemize}
	\item Como usuario general quiero loguearme al sistema web usando mail y contraseña.
	\newline Ponderación: 1.
	\newline Priorización: 2.
	\item Como usuario general quiero desloguearme del sistema web.
	\newline Ponderación: 1.
	\newline Priorización: 2.
	\item Como usuario general del sistema quiero recuperar y cambiar mi contraseña.
	\newline Ponderación: 3.
	\newline Priorización: 2.
	\item Como usuario general del sistema quiero ver y editar mi información de perfil.
	\newline Ponderación: 2.
	\newline Priorización: 3.

	\item Como administrador quiero dar de alta usuarios de tipo responsable y visitante.
	\newline Ponderación: 2.
	\newline Priorización: 2.
	\item Como administrador quiero modificar o eliminar usuarios de tipo responsable y visitante.
	\newline Ponderación: 2.
	\newline Priorización: 2.
	\item Como administrador quiero dar de alta una huerta y asociarla a un responsable.
	\newline Ponderación: 5.
	\newline Priorización: 1.
	\item Como administrador quiero modificar y eliminar cualquier huerta del sistema.
	\newline Ponderación: 3.
	\newline Priorización: 2.	
	\item Como administrador quiero asociar un sector, de una huerta, a un dispositivo usando su código único.
	\newline Ponderación: 3.
	\newline Priorización: 1.
	\item Como administrador quiero crear y editar los porcentajes de aceptación de humedad por sector en cualquier huerta.
	\newline Ponderación: 3.
	\newline Priorización: 1.
	\item Como administrador quiero ver y modificar los datos básicos de los roles del sistema.
	\newline Ponderación: 2.
	\newline Priorización: 3.
	\item Como administrador quiero visualizar las mediciones de un sector de cualquier huerta.
	\newline Ponderación: 5.
	\newline Priorización: 2.
	
	\item Como responsable quiero darme de alta en el sistema con mail y password.
	\newline Ponderación: 2.
	\newline Priorización: 1.
	\item Como responsable quiero dar de alta una huerta y que se asocie a mi usuario.
	\newline Ponderación: 3.
	\newline Priorización: 1.
	\item Como responsable quiero ver, modificar y eliminar mis huertas asociadas.
	\newline Ponderación: 3.
	\newline Priorización: 2.
	\item Como responsable quiero asociar un sector, de una mis huertas, a un dispositivo usando su código único.
	\newline Ponderación: 1.
	\newline Priorización: 1.
	\item Como responsable quiero administrar los porcentajes de aceptación de humedad por sector en mis huertas.
	\newline Ponderación: 1.
	\newline Priorización: 1.
	\item Como responsable quiero visualizar las mediciones de un sector de cualquiera de mis huertas.
	\newline Ponderación: 1.
	\newline Priorización: 1.
	\item Como responsable quiero crear usuarios con rol visitante y asociarlos a cualquiera de mis huertas.
	\newline Ponderación: 2.
	\newline Priorización: 3.
	
	\item Como usuario visitante quiero visualizar las huertas a las que estoy asociado.
	\newline Ponderación: 1.
	\newline Priorización: 1.
	\item Como usuario visitante quiero visualizar las métricas de un sector de cualquiera de las huertas a las que estoy asociado.
	\newline Ponderación: 1.
	\newline Priorización: 1.
\end{itemize}


\section{8. Entregables principales del proyecto}
\label{sec:entregables}

\begin{itemize}
	\item Manual de uso.
	\item Diagrama de los esquemático del sistema global.
	\item Diagrama de los componentes de un dispositivo.
	\item Diagrama de clases del software.
	\item Manual de armado e instalación del dispositivo.
	\item Repositorio con el código fuente.
	\item Informe final.
\end{itemize}

\section{9. Desglose del trabajo en tareas}
\label{sec:wbs}

\begin{enumerate}
\item Planificación del proyecto. (75 hs)
	\begin{enumerate}
	\item Definición de equipos y tareas. (5 hs)
	\item Definición del comportamiento del dispositivo. (20 hs)
	\item Definición del comportamiento del software. (40 hs)
	\item Definición de la integración entre el dispositivo y el software. (10 hs)
	\end{enumerate}
	
\item Investigación y armado del prototipo del dispositivo. (140 hs)
	\begin{enumerate}
	\item Investigación de componentes. (10 hs)
	\item Integración de la placa ESP32 con los sensores de humedad. (10 hs)
	\item Montado del dispositivo. (20 hs)
	\item Programación del dispositivo. (60 hs)
	\item Programación para apertura y cierre de la válvula de agua. (20 hs)
	\item Pruebas de integración. (20 hs)
	\end{enumerate}
	
\item Desarrollo del back-end del software. (215 hs)
	\begin{enumerate}
	\item Modelado del diagrama de clases. (20 hs)
	\item Desarrollo a arquitectura base del back-end. (15 hs)
	\item Desarrollo de endpoints de gestión de usuarios y roles. (40 hs)
	\item Desarrollo de endpoints de gestión permisos. (20 hs)
	\item Desarrollo de endpoints de alta, baja y modificación de huertas. (60 hs)
	\item Desarrollo de endpoints de administración de una huerta. (20 hs)
	\item Desarrollo de endpoints de visualización de métricas. (20 hs)
	\item Desarrollo de endpoints de integración con dispositivos. (40 hs)
	\end{enumerate}
	
\item Desarrollo del front-end del software. (255 hs)
	\begin{enumerate}
	\item Maquetado de las vistas generales del sitio web con la definición de estilos y colores. (40 hs)
	\item Desarrollo de la arquitectura base del front-end. (15 hs)
	\item Desarrollo de la vista gestión de usuarios y roles. (40 hs)
	\item Desarrollo de las vistas de alta, baja y modificación de huertas. (60 hs)
	\item Desarrollo de las vistas de administración de una huerta. (20 hs)
	\item Desarrollo de la vista para visualización de métricas. (80 hs)
	\end{enumerate}
\end{enumerate}

Cantidad total: 685 hs.

\section{10. Diagrama de Activity On Node}
\label{sec:AoN}

\begin{consigna}{red}
Armar el AoN a partir del WBS definido en la etapa anterior. 

%La figura \ref{fig:AoN} fue elaborada con el paquete latex tikz y pueden consultar la siguiente referencia \textit{online}:

%\url{https://www.overleaf.com/learn/latex/LaTeX_Graphics_using_TikZ:_A_Tutorial_for_Beginners_(Part_3)\%E2\%80\%94Creating_Flowcharts}

\end{consigna}

\begin{figure}[htpb]
\centering 
\includegraphics[width=.8\textwidth]{./Figuras/AoN.png}
\caption{Diagrama en \textit{Activity on Node}}
\label{fig:AoN}
\end{figure}

Indicar claramente en qué unidades están expresados los tiempos.
De ser necesario indicar los caminos semicríticos y analizar sus tiempos mediante un cuadro.
Es recomendable usar colores y un cuadro indicativo describiendo qué representa cada color, como se muestra en el siguiente ejemplo:



\section{11. Diagrama de Gantt}
\label{sec:gantt}

\begin{consigna}{red}

Existen muchos programas y recursos \textit{online} para hacer diagramas de gantt, entre los cuales destacamos:

\begin{itemize}
\item Planner
\item GanttProject
\item Trello + \textit{plugins}. En el siguiente link hay un tutorial oficial: \\ \url{https://blog.trello.com/es/diagrama-de-gantt-de-un-proyecto}
\item Creately, herramienta online colaborativa. \\\url{https://creately.com/diagram/example/ieb3p3ml/LaTeX}
\item Se puede hacer en latex con el paquete \textit{pgfgantt}\\ \url{http://ctan.dcc.uchile.cl/graphics/pgf/contrib/pgfgantt/pgfgantt.pdf}
\end{itemize}

Pegar acá una captura de pantalla del diagrama de Gantt, cuidando que la letra sea suficientemente grande como para ser legible. 
Si el diagrama queda demasiado ancho, se puede pegar primero la ``tabla'' del Gantt y luego pegar la parte del diagrama de barras del diagrama de Gantt.

Configurar el software para que en la parte de la tabla muestre los códigos del EDT (WBS).\\
Configurar el software para que al lado de cada barra muestre el nombre de cada tarea.\\
Revisar que la fecha de finalización coincida con lo indicado en el Acta Constitutiva.

En la figura \ref{fig:gantt}, se muestra un ejemplo de diagrama de gantt realizado con el paquete de \textit{pgfgantt}. En la plantilla pueden ver el código que lo genera y usarlo de base para construir el propio.

\begin{figure}[htbp]
\begin{center}
\begin{ganttchart}{1}{12}
  \gantttitle{2020}{12} \\
  \gantttitlelist{1,...,12}{1} \\
  \ganttgroup{Group 1}{1}{7} \\
  \ganttbar{Task 1}{1}{2} \\
  \ganttlinkedbar{Task 2}{3}{7} \ganttnewline
  \ganttmilestone{Milestone o hito}{7} \ganttnewline
  \ganttbar{Final Task}{8}{12}
  \ganttlink{elem2}{elem3}
  \ganttlink{elem3}{elem4}
\end{ganttchart}
\end{center}
\caption{Diagrama de gantt de ejemplo}
\label{fig:gantt}
\end{figure}


\begin{landscape}
\begin{figure}[htpb]
\centering 
\includegraphics[height=.85\textheight]{./Figuras/Gantt-2.png}
\caption{Ejemplo de diagrama de Gantt rotado}
\label{fig:diagGantt}
\end{figure}

\end{landscape}

\end{consigna}


\section{12. Presupuesto detallado del proyecto}
\label{sec:presupuesto}

\begin{consigna}{red}
Si el proyecto es complejo entonces separarlo en partes:
\begin{itemize}
	\item Un total global, indicando el subtotal acumulado por cada una de las áreas.
	\item El desglose detallado del subtotal de cada una de las áreas.
\end{itemize}

IMPORTANTE: No olvidarse de considerar los COSTOS INDIRECTOS.

\end{consigna}

\begin{table}[htpb]
\centering
\begin{tabularx}{\linewidth}{@{}|X|c|r|r|@{}}
\hline
\rowcolor[HTML]{C0C0C0} 
\multicolumn{4}{|c|}{\cellcolor[HTML]{C0C0C0}COSTOS DIRECTOS} \\ \hline
\rowcolor[HTML]{C0C0C0} 
Descripción &
  \multicolumn{1}{c|}{\cellcolor[HTML]{C0C0C0}Cantidad} &
  \multicolumn{1}{c|}{\cellcolor[HTML]{C0C0C0}Valor unitario} &
  \multicolumn{1}{c|}{\cellcolor[HTML]{C0C0C0}Valor total} \\ \hline
 &
  \multicolumn{1}{c|}{} &
  \multicolumn{1}{c|}{} &
  \multicolumn{1}{c|}{} \\ \hline
 &
  \multicolumn{1}{c|}{} &
  \multicolumn{1}{c|}{} &
  \multicolumn{1}{c|}{} \\ \hline
\multicolumn{1}{|l|}{} &
   &
   &
   \\ \hline
\multicolumn{1}{|l|}{} &
   &
   &
   \\ \hline
\multicolumn{3}{|c|}{SUBTOTAL} &
  \multicolumn{1}{c|}{} \\ \hline
\rowcolor[HTML]{C0C0C0} 
\multicolumn{4}{|c|}{\cellcolor[HTML]{C0C0C0}COSTOS INDIRECTOS} \\ \hline
\rowcolor[HTML]{C0C0C0} 
Descripción &
  \multicolumn{1}{c|}{\cellcolor[HTML]{C0C0C0}Cantidad} &
  \multicolumn{1}{c|}{\cellcolor[HTML]{C0C0C0}Valor unitario} &
  \multicolumn{1}{c|}{\cellcolor[HTML]{C0C0C0}Valor total} \\ \hline
\multicolumn{1}{|l|}{} &
   &
   &
   \\ \hline
\multicolumn{1}{|l|}{} &
   &
   &
   \\ \hline
\multicolumn{1}{|l|}{} &
   &
   &
   \\ \hline
\multicolumn{3}{|c|}{SUBTOTAL} &
  \multicolumn{1}{c|}{} \\ \hline
\rowcolor[HTML]{C0C0C0}
\multicolumn{3}{|c|}{TOTAL} &
   \\ \hline
\end{tabularx}%
\end{table}


\section{13. Gestión de riesgos}
\label{sec:riesgos}

\begin{consigna}{red}
a) Identificación de los riesgos (al menos cinco) y estimación de sus consecuencias:
 
Riesgo 1: detallar el riesgo (riesgo es algo que si ocurre altera los planes previstos de forma negativa)
\begin{itemize}
	\item Severidad (S): mientras más severo, más alto es el número (usar números del 1 al 10).\\
	Justificar el motivo por el cual se asigna determinado número de severidad (S).
	\item Probabilidad de ocurrencia (O): mientras más probable, más alto es el número (usar del 1 al 10).\\
	Justificar el motivo por el cual se asigna determinado número de (O). 
\end{itemize}   

Riesgo 2:
\begin{itemize}
	\item Severidad (S): 
	\item Ocurrencia (O):
\end{itemize}

Riesgo 3:
\begin{itemize}
	\item Severidad (S): 
	\item Ocurrencia (O):
\end{itemize}


b) Tabla de gestión de riesgos:      (El RPN se calcula como RPN=SxO)

\begin{table}[htpb]
\centering
\begin{tabularx}{\linewidth}{@{}|X|c|c|c|c|c|c|@{}}
\hline
\rowcolor[HTML]{C0C0C0} 
Riesgo & S & O & RPN & S* & O* & RPN* \\ \hline
       &   &   &     &    &    &      \\ \hline
       &   &   &     &    &    &      \\ \hline
       &   &   &     &    &    &      \\ \hline
       &   &   &     &    &    &      \\ \hline
       &   &   &     &    &    &      \\ \hline
\end{tabularx}%
\end{table}

Criterio adoptado: 
Se tomarán medidas de mitigación en los riesgos cuyos números de RPN sean mayores a...

Nota: los valores marcados con (*) en la tabla corresponden luego de haber aplicado la mitigación.

c) Plan de mitigación de los riesgos que originalmente excedían el RPN máximo establecido:
 
Riesgo 1: plan de mitigación (si por el RPN fuera necesario elaborar un plan de mitigación).
  Nueva asignación de S y O, con su respectiva justificación:
  - Severidad (S): mientras más severo, más alto es el número (usar números del 1 al 10).
          Justificar el motivo por el cual se asigna determinado número de severidad (S).
  - Probabilidad de ocurrencia (O): mientras más probable, más alto es el número (usar del 1 al 10).
          Justificar el motivo por el cual se asigna determinado número de (O).

Riesgo 2: plan de mitigación (si por el RPN fuera necesario elaborar un plan de mitigación).
 
Riesgo 3: plan de mitigación (si por el RPN fuera necesario elaborar un plan de mitigación).

\end{consigna}


\section{14. Gestión de la calidad}
\label{sec:calidad}

\begin{consigna}{red}
Para cada uno de los requerimientos del proyecto indique:
\begin{itemize} 
\item Req \#1: copiar acá el requerimiento.

\begin{itemize}
	\item Verificación para confirmar si se cumplió con lo requerido antes de mostrar el sistema al cliente. Detallar 
	\item Validación con el cliente para confirmar que está de acuerdo en que se cumplió con lo requerido. Detallar  
\end{itemize}

\end{itemize}

Tener en cuenta que en este contexto se pueden mencionar simulaciones, cálculos, revisión de hojas de datos, consulta con expertos, mediciones, etc.  Las acciones de verificación suelen considerar al entregable como ``caja blanca'', es decir se conoce en profundidad su funcionamiento interno.  En cambio, las acciones de validación suelen considerar al entregable como ``caja negra'', es decir, que no se conocen los detalles de su funcionamiento interno.

\end{consigna}

\section{15. Procesos de cierre}    
\label{sec:cierre}

\begin{consigna}{red}
Establecer las pautas de trabajo para realizar una reunión final de evaluación del proyecto, tal que contemple las siguientes actividades:

\begin{itemize}
	\item Pautas de trabajo que se seguirán para analizar si se respetó el Plan de Proyecto original:
	 - Indicar quién se ocupará de hacer esto y cuál será el procedimiento a aplicar. 
	\item Identificación de las técnicas y procedimientos útiles e inútiles que se emplearon, y los problemas que surgieron y cómo se solucionaron:
	 - Indicar quién se ocupará de hacer esto y cuál será el procedimiento para dejar registro.
	\item Indicar quién organizará el acto de agradecimiento a todos los interesados, y en especial al equipo de trabajo y colaboradores:
	  - Indicar esto y quién financiará los gastos correspondientes.
\end{itemize}

\end{consigna}


\end{document}
