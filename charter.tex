\documentclass[11pt]{charter}

% El títulos de la memoria, se usa en la carátula y se puede usar el cualquier lugar del documento con el comando \ttitle
\titulo{Título del proyecto} 

% Nombre del posgrado, se usa en la carátula y se puede usar el cualquier lugar del documento con el comando \degreename
\posgrado{Carrera de Especialización en Sistemas Embebidos} 
%\posgrado{Carrera de Especialización en Internet de las Cosas} 
%\posgrado{Maestría en Sistemas Embebidos} 
%\posgrado{Maestría en Internet de las cosas}

% Tu nombre, se puede usar el cualquier lugar del documento con el comando \authorname
\autor{Nombre del Autor} 

% El nombre del director y co-director, se puede usar el cualquier lugar del documento con el comando \supname y \cosupname
\director{Nombre del Director (pertenencia)} 
\codirector{} % si queda vacio no se incluye

% Nombre del cliente, quien va a aprobar los resultados del proyecto, se puede usar con el comando \clientename
\cliente{Nombre del cliente}

% Nombre y pertenencia de los jurados, se pueden usar el cualquier lugar del documento con el comando \jur1name, \jur2name y \jur3name
\juradoUNO{Nombre del jurado 1 (pertenencia)} 
\juradoDOS{Nombre del jurado 2 (pertenencia)}
\juradoTRES{Nombre del jurado 3 (pertenencia)}
 
\fechaINICIO{22 de junio de 2020}				%Fecha de inicio de la cursada de GdP
\fechaFINALPlanificacion{22 de Agosto de 2020} 	%Fecha de final de cursada de GdP
\fechaFINALTrabajo{22 de diciembre de 2020}		%Fecha de defensa pública del trabajo final


%\usepackage{geometry}
%


\begin{document}

\maketitle
\thispagestyle{empty}
\pagebreak


\thispagestyle{empty}
\tableofcontents{}
\pagebreak


\section{Registros de cambios}
\label{sec:registro}


\begin{table}[ht]
\label{tab:registro}
\centering

\begin{tabularx}{\linewidth}{@{}|c|X|c|@{}}
\hline
\rowcolor[HTML]{C0C0C0} 
Revisión & \multicolumn{1}{c|}{\cellcolor[HTML]{C0C0C0}Detalles de los cambios realizados} & Fecha      \\ \hline
1.0      & Creación del documento                                                          & 22/06/2020 \\ \hline
1.1      & Ejemplo de un texto muy largo que debiera ocupar más de una línea para que tengan de ejemplo                                                                                																						   & dd/mm/aaaa \\ \hline
1.2      & Otro ejemplo \newline
		   Con texto partido \newline
		   En varias líneas \newline
		   A propósito                                                                     & dd/mm/aaaa       \\ \hline
\end{tabularx}%
\end{table}

\pagebreak



\section{Acta de Constitución del Proyecto}
\label{sec:acta}

\begin{flushright}
Buenos Aires, \fechaINICIOname
\end{flushright}

\vspace{2cm}

Por medio de la presente se acuerda con el Ing. \authorname\hspace{1px} que su Trabajo Final de la \degreename\hspace{1px} se titulará ``\ttitle'', consistirá esencialmente en el prototipo preliminar de un ........ , y tendrá un presupuesto preliminar estimado de 600 hs de trabajo y \$XXX, con fecha de inicio \fechaINICIOname\hspace{1px} y fecha de presentación pública \fechaFINALname.

Se adjunta a esta acta la planificación inicial.

\vfill

%\begin{tabularx}{\linewidth}{@{}XXX@{}}
%
%\centering\rule[-1ex]{0pt}{3cm} Ariel Lutenberg\newline Director posgrado FIUBA &  & Nombre del cliente \newline Empresa del cliente \\ 
%\centering\rule[-1ex]{0pt}{3cm}      											& \supname \newline Director del Trabajo Final 		& 	 \\ 
%\centering\rule[-1ex]{0pt}{3cm}\jurunoname \newline Jurado del Trabajo Final&  & \jurdosname \newline Jurado del Trabajo Final\\
%\centering\rule[-1ex]{0pt}{3cm}      											& \jurtresname \newline Jurado del Trabajo Final	& 	 \\ 
%\end{tabularx}%

\begin{table}[ht]
\centering
\begin{tabular}{ccc}
\begin{tabular}[c]{@{}c@{}}Ariel Lutenberg \\ Director posgrado FIUBA\end{tabular} &  & \begin{tabular}[c]{@{}c@{}}\clientename \\ Empresa del cliente\end{tabular} \vspace{2.5cm} \\ 
\multicolumn{3}{c}{\begin{tabular}[c]{@{}c@{}} \supname \\ Director del Trabajo Final\end{tabular}} \vspace{2.5cm} \\
\begin{tabular}[c]{@{}c@{}}\jurunoname \\ Jurado del Trabajo Final\end{tabular}     &  & \begin{tabular}[c]{@{}c@{}}\jurdosname\\ Jurado del Trabajo Final\end{tabular}  \vspace{2.5cm}  \\
\multicolumn{3}{c}{\begin{tabular}[c]{@{}c@{}} \jurtresname\\ Jurado del Trabajo Final\end{tabular}} \vspace{.5cm}                                                                     
\end{tabular}

\end{table}

\section{Descripción técnica-conceptual del Proyecto a realizar}
\section{Identificación y análisis de los interesados}
\section{1. Propósito del proyecto}
\section{2. Alcance del proyecto}
\section{3. Supuestos del proyecto}
\section{4. Requerimientos}
\section{5. Entregables principales del proyecto}
\section{6. Desglose del trabajo en tareas}
\section{7. Diagrama de Activity On Node}
\section{8. Diagrama de Gantt}
\section{9. Matriz de uso de recursos de materiales}
\section{10. Presupuesto detallado del proyecto}
\section{11. Matriz de asignación de responsabilidades}
\section{12. Gestión de riesgos}
\section{13. Gestión de la calidad}
\section{14. Comunicación del proyecto}
\section{15. Gestión de Compras}
\section{16. Seguimiento y control}
\section{17. Procesos de cierre}    

%
%\section*{Project Roles}
%\label{sec:roles}
%\begin{table}[h]
%  \label{tbl:roles}
%  \begin{center}
%    \begin{tabular}{c|c}
%      Name & Role\\\hline
%      YOUR NAME HERE & Project Manager\\\hline
%      & Sponsor\\\hline
%      & Customer / User Contact\\\hline
%      & Team Members\\\hline
%      & Stakeholders\\\hline
%    \end{tabular}
%  \end{center}
%\end{table}



\end{document}
