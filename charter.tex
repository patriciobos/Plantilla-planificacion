\documentclass[
11pt, % The default document font size, options: 10pt, 11pt, 12pt
%codirector, % Uncomment to add a codirector to the title page
]{charter} 




% El títulos de la memoria, se usa en la carátula y se puede usar el cualquier lugar del documento con el comando \ttitle
\titulo{Desarrollo de aplicación para empleados SER\&PRO Services \& Products S.A. con notificación push} 

% Nombre del posgrado, se usa en la carátula y se puede usar el cualquier lugar del documento con el comando \degreename
%\posgrado{Carrera de Especialización en Sistemas Embebidos} 
\posgrado{Carrera de Especialización en Internet de las Cosas} 
%\posgrado{Carrera de Especialización en Intelegencia Artificial}
%\posgrado{Maestría en Sistemas Embebidos} 
%\posgrado{Maestría en Internet de las cosas}

% Tu nombre, se puede usar el cualquier lugar del documento con el comando \authorname
\autor{Ing. 	Fabián Alejandro Banderas Benítez} 

% El nombre del director y co-director, se puede usar el cualquier lugar del documento con el comando \supname y \cosupname y \pertesupname y \pertecosupname
\director{Mg. Ing. Yoel Yamil López}
\pertenenciaDirector{FIUBA} 
% FIXME:NO IMPLEMENTADO EL CODIRECTOR ni su pertenencia
%\codirector{Codirector a definir} % para que aparezca en la portada se debe descomentar la opción codirector en el documentclass
\pertenenciaCoDirector{FIUBA}

% Nombre del cliente, quien va a aprobar los resultados del proyecto, se puede usar con el comando \clientename y \empclientename
\cliente{Ligia Geomar Delli Valladares}
\empresaCliente{SER\&PRO Services \& Products S.A.}

% Nombre y pertenencia de los jurados, se pueden usar el cualquier lugar del documento con el comando \jurunoname, \jurdosname y \jurtresname y \perteunoname, \pertedosname y \pertetresname.
\juradoUno{Jurado a definir (1)}
\pertenenciaJurUno{FIUBA (1)} 
\juradoDos{Jurado a definir (2)}
\pertenenciaJurDos{FIUBA (2)}
\juradoTres{Jurado a definir (3)}
\pertenenciaJurTres{FIUBA (3)}
 
\fechaINICIO{21 de octubre de 2022}		%Fecha de inicio de la cursada de GdP \fechaInicioName
\fechaFINALPlan{8 de diciembre de 2022} 	%Fecha de final de cursada de GdP
\fechaFINALTrabajo{8 de diciembre de 2022}	%Fecha de defensa pública del trabajo final





\begin{document}

\maketitle
\thispagestyle{empty}
\pagebreak


\thispagestyle{empty}
{\setlength{\parskip}{0pt}
\tableofcontents{}
}
\pagebreak


\section*{Registros de cambios}
\label{sec:registro}


\begin{table}[ht]
\label{tab:registro}
\centering
\begin{tabularx}{\linewidth}{@{}|c|X|c|@{}}
\hline
\rowcolor[HTML]{C0C0C0} 
Revisión & \multicolumn{1}{c|}{\cellcolor[HTML]{C0C0C0}Detalles de los cambios realizados} & Fecha      \\ \hline
0      & Creación del documento                                 &\fechaInicioName \\ \hline
1      & Se completa hasta el punto 5 inclusive                 & 3 de noviembre de 2022 \\ \hline
2      & Se completa hasta el punto 8 inclusive                 & 10 de noviembre de 2022 \\ \hline
3      & Se completa hasta el punto 12 inclusive                 & 20 de noviembre de 2022 \\ \hline
4      & Se completa hasta el punto 17 inclusive                 & 23 de noviembre de 2022 \\ \hline
%		  Se puede agregar algo más \newline
%		  En distintas líneas \newline
%		  Así                                                    & dd/mm/aaaa \\ \hline
%3      & Se completa hasta el punto 11 inclusive                & dd/mm/aaaa \\ \hline
%4      & Se completa el plan	                                 & dd/mm/aaaa \\ \hline
\end{tabularx}
\end{table}

\pagebreak



\section*{Acta de constitución del proyecto}
\label{sec:acta}

\begin{flushright}
Guayaquil, \fechaInicioName

\end{flushright}

\vspace{2cm}

Por medio de la presente se acuerda con el \authorname\hspace{1px} que su Trabajo Final de la \degreename\hspace{1px} se titulará ``\ttitle'', consistirá esencialmente en {la implementación de un prototipo de un sistema de control de empleados}, y tendrá un presupuesto preliminar estimado de {600} hs de trabajo y {\$3764}, con fecha de inicio \fechaInicioName\hspace{1px} y fecha de presentación pública \fechaFinalName.

Se adjunta a esta acta la planificación inicial.

\vfill

% Esta parte se construye sola con la información que hayan cargado en el preámbulo del documento y no debe modificarla
\begin{table}[ht]
\centering
\begin{tabular}{ccc}
\begin{tabular}[c]{@{}c@{}}Dr. Ing. Ariel Lutenberg \\ Director posgrado FIUBA\end{tabular} & \hspace{2cm} & \begin{tabular}[c]{@{}c@{}}Ing. \clientename \\ \empclientename \end{tabular} \vspace{2.5cm} \\ 
\multicolumn{3}{c}{\begin{tabular}[c]{@{}c@{}} \supname \\ Director del Trabajo Final\end{tabular}} \vspace{2.5cm} \\
%\begin{tabular}[c]{@{}c@{}}\jurunoname \\ Jurado del Trabajo Final\end{tabular}     &  & \begin{tabular}[c]{@{}c@{}}\jurdosname\\ Jurado del Trabajo Final\end{tabular}  \vspace{2.5cm}  \\
%\multicolumn{3}{c}{\begin{tabular}[c]{@{}c@{}} \jurtresname\\ Jurado del Trabajo Final\end{tabular}} \vspace{.5cm}                                                                     
\end{tabular}
\end{table}




\section{1. Descripción técnica-conceptual del proyecto a realizar}
\label{sec:descripcion}


\begin{consigna}{black} % El bloque "consigna" se usa para poner texto en rojo y dar una pequeña ayuda sobre cómo completar la sección

Debido a la incursión de medios Smartphone, dispositivos electrónicos que se conectan a través de internet, es necesario dar una solución más sencilla para reportes de registro de entrada y salida del personal que labora en el interior de la empresa. ``En la Figura \ref{fig:diagBloques} se presenta el diagrama en bloques del sistema. Se observa que desde el dispositivo biométrico se hace el registro, este se procesa, valida, asigna de forma interna para luego extraer  enviar la notificación respectiva a quien corresponda.

A falta de un registro de entrada y salida con notificaciones para crear una mejor distribución de tiempos entre empleados se presenta la propuesta que consta de:

\begin{itemize}
	\item Supervisor
	\item Empleados
	\item Destinatarios para recibir notificación
	\item Registros
\end{itemize}

Los dispositivos a través los cuales se generan los registros, se conectarán para enviar los datos e intercambiarlos con los diferentes dispositivos. Los requerimientos mínimos se muestran a continuación:

\begin{itemize}
	\item La aplicación permite la autenticación de los miembros registrados.
	\item El empleado hará el registro de ingreso.
	\item El supervisor y personal recibirá la notificación push de ingreso.
	\item El cambio de estado y el tiempo de estancia empezará.
	\item El empleado hará el registro de salida.
	\item El supervisor y personal recibirá la notificación push de salida.
	\item Las métricas de cada uno de los empleados deben ser visuales a través de grafos.
\end{itemize}

%\vspace{25px}

\begin{figure}[htpb]
\centering 
\includegraphics[width=0.85\textwidth]{./Figuras/diagBloques.png}
\caption{Diagrama en bloques del sistema.}
\label{fig:diagBloques}
\end{figure}

\vspace{25px}

\end{consigna}


\section{2. Identificación y análisis de los interesados}
\label{sec:interesados}



\begin{table}[ht]
%\caption{Identificación de los interesados}
%\label{tab:interesados}
\begin{tabularx}{\linewidth}{@{}|l|X|X|l|@{}}
\hline
\rowcolor[HTML]{C0C0C0} 
Rol           & Nombre y Apellido & Organización 	& Puesto 	\\ \hline
Auspiciante   & Ing. \clientename      &\empclientename	& Gerente  	\\ \hline
Cliente       & Ing. \clientename      &\empclientename	& Gerente  	\\ \hline
Impulsor      & Ing. Flavio Bolívar Vinueza Barzola \newline
Ing. Dennys Alejandro Montero Huilca &\empclientename	& Supervisor\\ \hline
Responsable   & \authorname       & FIUBA        	& Alumno 	\\ \hline
Colaboradores & Sr. Maike Rafael Alvarado Melendez \newline
Sr. Denny Alberto Cuzme Morales &\empclientename 	& Producción\\ \hline
Orientador    & \supname	      & \pertesupname 	& Director Trabajo final \\ \hline
Equipo        & Ing. Flavio Bolívar Vinueza Barzola \newline
				Ing. Dennys Alejandro Montero Huilca \newline 
				Sr. Maike Rafael Alvarado Melendez \newline
				Sr. Denny Alberto Cuzme Morales & \empclientename	& Empleados \\ \hline
Opositores    & Sra. Grace Ivonne Maridueña Carlier & Serintu S.A. & Gerente General \\ \hline
Usuario final & Ing. Gabriela Salvador & \empclientename	&    RRHH   	\\ \hline
\end{tabularx}
\end{table}



\section{3. Propósito del proyecto}
\label{sec:proposito}



\begin{consigna}{black}
 ``El propósito de este proyecto es mostrar mensajes informativos desde la aplicación para los usuarios con llamadas a la acción personalizadas con el fin de comunicar al usuario final.''.
 
\end{consigna}

\section{4. Alcance del proyecto}
\label{sec:alcance}




El presente proyecto contempla la notificación de los registros, entrada,
tiempo, duración y salida del mes de cada empleado en la empresa \empclientename, con el uso de una herramienta digital. Para el desarrollo del proyecto se hace uso de la metodología SCRUM.

El presente proyecto no incluye el mantenimiento de la infraestructura digital y física del
aplicativo.




\section{5. Supuestos del proyecto}
\label{sec:supuestos}



Para el desarrollo del presente proyecto se supone que: 

\begin{itemize}
	\item La disponibilidad del cliente Ing. \clientename, impulsores o colaboradores encargados para la guía en el avance del proyecto. 
	\item Los recursos actuales pueden ser modernizados por necesidad propia de la empresa.
	\item Cambios o implementación de leyes para regulación nacional para el uso de los equipos.
	\item Cambios del personal de la empresa.
	\item Cambios en directrices de la empresa.
	\item Prioridades de la empresa por eventos inesperados.
	\item Fenómenos naturales.
	\item Daño del lector biométrico LX50 ZK.
	\item Corte de energía.
\end{itemize}



\section{6. Requerimientos}
\label{sec:requerimientos}


\begin{enumerate}
	\item Requerimientos funcionales
		\begin{enumerate}
			\item El sistema debe autenticar solo a los empleados de planta de la empresa registrados.
			\item El empleado se registra en el sistema.
			\item El supervisor y empleado reciben una notificación push de ingreso al sistema.
			\item El supervisor y usuario pueden ver el tiempo de estancia que tuvo un empleado en la empresa en un determinado tiempo.
			\item El supervisor y empleado reciben una notificación push del reporte ingresos a la empresa.
			\item El supervisor y usuario reciben una notificación push del reporte de salida de la empresa.
			\item El supervisor tiene acceso al reporte de tiempo promedio promedio del empleado en la empresa en un rango de fechas. 
			\item Cada empleado tiene un registro de entradas, salidas con tiempos de estancia que se pueden ver a través de grafos.
			\item El empleado puede ver el tiempo de estancia total  que tuvo en un rango de fechas desde realizó el registro de usuario en el sistema.
		\end{enumerate}
	\item Requerimientos de documentación
		\begin{enumerate}
			\item Manual de usuario.
			\item Planilla de casos de uso.
		\end{enumerate}
	\item Requerimiento de testing
		\begin{enumerate}
			\item Validación de datos para el registro del sistema.
			\item Notificación push para el empleado.
			\item Notificación push para el supervisor.
		\end{enumerate}
	\item Requerimientos de la interfaz.
			\begin{enumerate}
			\item Debe contar con los distintivos de la empresa y colores de marca.
			\item Usar técnica de ingeniería de software para el diseño de la interfaz de usuario.
		\end{enumerate}
	\item Requerimientos interoperabilidad.
			\begin{enumerate}
			\item El usuario puede ingresar a su perfil de empleado con credenciales únicas.
			\item El supervisor puede visualizar métricas de todos los empleados.
			\item Realizar la evaluación de usabilidad de la aplicación mediante el uso de la Norma ISO/IEC 25010.
		\end{enumerate}
\end{enumerate}



\section{7. Historias de usuarios (\textit{Product backlog})}
\label{sec:backlog}


Para realizar la estimación del proyecto se emplea la técnica de T-shirt, que se describe en la siguiente tabla.

\begin{table}[h!]
%\caption{T-shirt}
%\label{tab:puntos estimados - horas de trabajo}
\centering
 \begin{tabular}{|c c c|} 
\hline
\rowcolor[HTML]{C0C0C0} 
Medida & Puntos estimados & Horas de trabajo	\\ \hline
XS     & 1               & 10  	  \\ \hline
S      & 2               & 10  	  \\ \hline
M      & 3               & 20    \\ \hline
L      & 5               & 40    \\ \hline
XL     & 8               & 60      \\ \hline
%\end{tabularx}
\end{tabular}
\end{table}


\section{8. Entregables principales del proyecto}
\label{sec:entregables}



Los entregables del proyecto son:

\begin{itemize}
	\item Finalizar lista de materiales.
	\item Crear cronograma de proyecto.
	\item Enviar invitaciones de calendario a los integrantes del grupo.
	\item Diseño de web \textit{assets}.
	\item Necesidades de contenido.
	\item Necesidades de animación de enlace.
	\item Manual de uso.
	\item Enviar encuesta de usabilidad a los empleados.
\end{itemize}



\section{9. Desglose del trabajo en tareas}
\label{sec:wbs}

\begin{consigna}{red}
El WBS debe tener relación directa o indirecta con los requerimientos.  Son todas las actividades que se harán en el proyecto para dar cumplimiento a los requerimientos. Se recomienda mostrar el WBS mediante una lista indexada:

\begin{enumerate}
\item Grupo de tareas 1
	\begin{enumerate}
	\item Tarea 1 (tantas hs)
	\item Tarea 2 (tantas hs)
	\item Tarea 3 (tantas hs)
	\end{enumerate}
\item Grupo de tareas 2
	\begin{enumerate}
	\item Tarea 1 (tantas hs)
	\item Tarea 2 (tantas hs)
	\item Tarea 3 (tantas hs)
	\end{enumerate}
\item Grupo de tareas 3
	\begin{enumerate}
	\item Tarea 1 (tantas hs)
	\item Tarea 2 (tantas hs)
	\item Tarea 3 (tantas hs)
	\item Tarea 4 (tantas hs)
	\item Tarea 5 (tantas hs)
	\end{enumerate}
\end{enumerate}

Cantidad total de horas: (tantas hs)

Se recomienda que no haya ninguna tarea que lleve más de 40 hs. 

\end{consigna}

\section{10. Diagrama de Activity On Node}
\label{sec:AoN}

\begin{consigna}{red}
Armar el AoN a partir del WBS definido en la etapa anterior. 

%La figura \ref{fig:AoN} fue elaborada con el paquete latex tikz y pueden consultar la siguiente referencia \textit{online}:

%\url{https://www.overleaf.com/learn/latex/LaTeX_Graphics_using_TikZ:_A_Tutorial_for_Beginners_(Part_3)\%E2\%80\%94Creating_Flowcharts}

\end{consigna}

\begin{figure}[htpb]
\centering 
\includegraphics[width=.8\textwidth]{./Figuras/AoN.png}
\caption{Diagrama en \textit{Activity on Node}}
\label{fig:AoN}
\end{figure}

Indicar claramente en qué unidades están expresados los tiempos.
De ser necesario indicar los caminos semicríticos y analizar sus tiempos mediante un cuadro.
Es recomendable usar colores y un cuadro indicativo describiendo qué representa cada color, como se muestra en el siguiente ejemplo:



\section{11. Diagrama de Gantt}
\label{sec:gantt}

\begin{consigna}{red}

Existen muchos programas y recursos \textit{online} para hacer diagramas de gantt, entre los cuales destacamos:

\begin{itemize}
\item Planner
\item GanttProject
\item Trello + \textit{plugins}. En el siguiente link hay un tutorial oficial: \\ \url{https://blog.trello.com/es/diagrama-de-gantt-de-un-proyecto}
\item Creately, herramienta online colaborativa. \\\url{https://creately.com/diagram/example/ieb3p3ml/LaTeX}
\item Se puede hacer en latex con el paquete \textit{pgfgantt}\\ \url{http://ctan.dcc.uchile.cl/graphics/pgf/contrib/pgfgantt/pgfgantt.pdf}
\end{itemize}

Pegar acá una captura de pantalla del diagrama de Gantt, cuidando que la letra sea suficientemente grande como para ser legible. 
Si el diagrama queda demasiado ancho, se puede pegar primero la ``tabla'' del Gantt y luego pegar la parte del diagrama de barras del diagrama de Gantt.

Configurar el software para que en la parte de la tabla muestre los códigos del EDT (WBS).\\
Configurar el software para que al lado de cada barra muestre el nombre de cada tarea.\\
Revisar que la fecha de finalización coincida con lo indicado en el Acta Constitutiva.

En la figura \ref{fig:gantt}, se muestra un ejemplo de diagrama de gantt realizado con el paquete de \textit{pgfgantt}. En la plantilla pueden ver el código que lo genera y usarlo de base para construir el propio.

\begin{figure}[htbp]
\begin{center}
\begin{ganttchart}{1}{12}
  \gantttitle{2020}{12} \\
  \gantttitlelist{1,...,12}{1} \\
  \ganttgroup{Group 1}{1}{7} \\
  \ganttbar{Task 1}{1}{2} \\
  \ganttlinkedbar{Task 2}{3}{7} \ganttnewline
  \ganttmilestone{Milestone o hito}{7} \ganttnewline
  \ganttbar{Final Task}{8}{12}
  \ganttlink{elem2}{elem3}
  \ganttlink{elem3}{elem4}
\end{ganttchart}
\end{center}
\caption{Diagrama de gantt de ejemplo}
\label{fig:gantt}
\end{figure}


\begin{landscape}
\begin{figure}[htpb]
\centering 
\includegraphics[height=.85\textheight]{./Figuras/Gantt-2.png}
\caption{Ejemplo de diagrama de Gantt rotado}
\label{fig:diagGantt}
\end{figure}

\end{landscape}

\end{consigna}


\section{12. Presupuesto detallado del proyecto}
\label{sec:presupuesto}

\begin{consigna}{red}
Si el proyecto es complejo entonces separarlo en partes:
\begin{itemize}
	\item Un total global, indicando el subtotal acumulado por cada una de las áreas.
	\item El desglose detallado del subtotal de cada una de las áreas.
\end{itemize}

IMPORTANTE: No olvidarse de considerar los COSTOS INDIRECTOS.

\end{consigna}

\begin{table}[htpb]
\centering
\begin{tabularx}{\linewidth}{@{}|X|c|r|r|@{}}
\hline
\rowcolor[HTML]{C0C0C0} 
\multicolumn{4}{|c|}{\cellcolor[HTML]{C0C0C0}COSTOS DIRECTOS} \\ \hline
\rowcolor[HTML]{C0C0C0} 
Descripción &
  \multicolumn{1}{c|}{\cellcolor[HTML]{C0C0C0}Cantidad} &
  \multicolumn{1}{c|}{\cellcolor[HTML]{C0C0C0}Valor unitario} &
  \multicolumn{1}{c|}{\cellcolor[HTML]{C0C0C0}Valor total} \\ \hline
 &
  \multicolumn{1}{c|}{} &
  \multicolumn{1}{c|}{} &
  \multicolumn{1}{c|}{} \\ \hline
 &
  \multicolumn{1}{c|}{} &
  \multicolumn{1}{c|}{} &
  \multicolumn{1}{c|}{} \\ \hline
\multicolumn{1}{|l|}{} &
   &
   &
   \\ \hline
\multicolumn{1}{|l|}{} &
   &
   &
   \\ \hline
\multicolumn{3}{|c|}{SUBTOTAL} &
  \multicolumn{1}{c|}{} \\ \hline
\rowcolor[HTML]{C0C0C0} 
\multicolumn{4}{|c|}{\cellcolor[HTML]{C0C0C0}COSTOS INDIRECTOS} \\ \hline
\rowcolor[HTML]{C0C0C0} 
Descripción &
  \multicolumn{1}{c|}{\cellcolor[HTML]{C0C0C0}Cantidad} &
  \multicolumn{1}{c|}{\cellcolor[HTML]{C0C0C0}Valor unitario} &
  \multicolumn{1}{c|}{\cellcolor[HTML]{C0C0C0}Valor total} \\ \hline
\multicolumn{1}{|l|}{} &
   &
   &
   \\ \hline
\multicolumn{1}{|l|}{} &
   &
   &
   \\ \hline
\multicolumn{1}{|l|}{} &
   &
   &
   \\ \hline
\multicolumn{3}{|c|}{SUBTOTAL} &
  \multicolumn{1}{c|}{} \\ \hline
\rowcolor[HTML]{C0C0C0}
\multicolumn{3}{|c|}{TOTAL} &
   \\ \hline
\end{tabularx}%
\end{table}


\section{13. Gestión de riesgos}
\label{sec:riesgos}

\begin{consigna}{red}
a) Identificación de los riesgos (al menos cinco) y estimación de sus consecuencias:
 
Riesgo 1: detallar el riesgo (riesgo es algo que si ocurre altera los planes previstos de forma negativa)
\begin{itemize}
	\item Severidad (S): mientras más severo, más alto es el número (usar números del 1 al 10).\\
	Justificar el motivo por el cual se asigna determinado número de severidad (S).
	\item Probabilidad de ocurrencia (O): mientras más probable, más alto es el número (usar del 1 al 10).\\
	Justificar el motivo por el cual se asigna determinado número de (O). 
\end{itemize}   

Riesgo 2:
\begin{itemize}
	\item Severidad (S): 
	\item Ocurrencia (O):
\end{itemize}

Riesgo 3:
\begin{itemize}
	\item Severidad (S): 
	\item Ocurrencia (O):
\end{itemize}


b) Tabla de gestión de riesgos:      (El RPN se calcula como RPN=SxO)

\begin{table}[htpb]
\centering
\begin{tabularx}{\linewidth}{@{}|X|c|c|c|c|c|c|@{}}
\hline
\rowcolor[HTML]{C0C0C0} 
Riesgo & S & O & RPN & S* & O* & RPN* \\ \hline
       &   &   &     &    &    &      \\ \hline
       &   &   &     &    &    &      \\ \hline
       &   &   &     &    &    &      \\ \hline
       &   &   &     &    &    &      \\ \hline
       &   &   &     &    &    &      \\ \hline
\end{tabularx}%
\end{table}

Criterio adoptado: 
Se tomarán medidas de mitigación en los riesgos cuyos números de RPN sean mayores a...

Nota: los valores marcados con (*) en la tabla corresponden luego de haber aplicado la mitigación.

c) Plan de mitigación de los riesgos que originalmente excedían el RPN máximo establecido:
 
Riesgo 1: plan de mitigación (si por el RPN fuera necesario elaborar un plan de mitigación).
  Nueva asignación de S y O, con su respectiva justificación:
  - Severidad (S): mientras más severo, más alto es el número (usar números del 1 al 10).
          Justificar el motivo por el cual se asigna determinado número de severidad (S).
  - Probabilidad de ocurrencia (O): mientras más probable, más alto es el número (usar del 1 al 10).
          Justificar el motivo por el cual se asigna determinado número de (O).

Riesgo 2: plan de mitigación (si por el RPN fuera necesario elaborar un plan de mitigación).
 
Riesgo 3: plan de mitigación (si por el RPN fuera necesario elaborar un plan de mitigación).

\end{consigna}


\section{14. Gestión de la calidad}
\label{sec:calidad}

\begin{consigna}{red}
Para cada uno de los requerimientos del proyecto indique:
\begin{itemize} 
\item Req \#1: copiar acá el requerimiento.

\begin{itemize}
	\item Verificación para confirmar si se cumplió con lo requerido antes de mostrar el sistema al cliente. Detallar 
	\item Validación con el cliente para confirmar que está de acuerdo en que se cumplió con lo requerido. Detallar  
\end{itemize}

\end{itemize}

Tener en cuenta que en este contexto se pueden mencionar simulaciones, cálculos, revisión de hojas de datos, consulta con expertos, mediciones, etc.  Las acciones de verificación suelen considerar al entregable como ``caja blanca'', es decir se conoce en profundidad su funcionamiento interno.  En cambio, las acciones de validación suelen considerar al entregable como ``caja negra'', es decir, que no se conocen los detalles de su funcionamiento interno.

\end{consigna}

\section{15. Procesos de cierre}    
\label{sec:cierre}

\begin{consigna}{red}
Establecer las pautas de trabajo para realizar una reunión final de evaluación del proyecto, tal que contemple las siguientes actividades:

\begin{itemize}
	\item Pautas de trabajo que se seguirán para analizar si se respetó el Plan de Proyecto original:
	 - Indicar quién se ocupará de hacer esto y cuál será el procedimiento a aplicar. 
	\item Identificación de las técnicas y procedimientos útiles e inútiles que se emplearon, y los problemas que surgieron y cómo se solucionaron:
	 - Indicar quién se ocupará de hacer esto y cuál será el procedimiento para dejar registro.
	\item Indicar quién organizará el acto de agradecimiento a todos los interesados, y en especial al equipo de trabajo y colaboradores:
	  - Indicar esto y quién financiará los gastos correspondientes.
\end{itemize}

\end{consigna}


\end{document}
